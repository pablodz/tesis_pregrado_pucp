%----------------------------------------------------------------------------------------
%	Tabla de contenidos
%----------------------------------------------------------------------------------------

%\newpage
%\clearpage{\pagestyle{empty}\cleardoublepage}
%\doublespacing
%\newpage

\newpage
\addcontentsline{toc}{chapter}{\contentsname}
\tableofcontents 
\newpage

\addcontentsline{toc}{chapter}{\listfigurename}
\listoffigures 
\newpage

\addcontentsline{toc}{chapter}{\listtablename}
\listoftables 
\newpage

\addcontentsline{toc}{chapter}{Índice de símbolos}

\begin{center}
	{\Large  Índice de símbolos}
\end{center}

Se utilizan las siguientes normas DIN para la definición y designación de símbolos\\


\begin{mytable}[H]
	%\footnotesize
	\centering
	%\caption{Pines necesarios en el microprocesador.}
	%\label{tab:pines necesarios en el microprocesador}
	\begin{tabular}{ll}
		\multicolumn{1}{c}{Concepto} & \multicolumn{1}{c}{Norma} \\
		Unidades & DIN 1301 \\
		Signos matemáticos & DIN 1302 \\
		Notación para fórmulas en general & SI 2019 \\
		Masa, peso, fuerza de peso, aceleración de caída & DIN 1305 \\
		Rotación. Hélice. Ángulo & DIN 1312 \\
		Densidad & DIN 1306 \\
		Presión & DIN 1314 \\
		Redondeado de números & DIN 1333 \\
		Estado normal Volumen normal & DIN 1343 \\
		Notación en resistencia de materiales & DIN 1350
	\end{tabular}
	%\begin{myflushcenteraftertable}	
	%	Fuente: Elaboración propia.
	%\end{myflushcenteraftertable}
\end{mytable}

Unidades de medida según DIN 1301


\newpage
