%----------------------------------------------------------------------------------------
%	Resumen
%----------------------------------------------------------------------------------------

\newpage
\clearpage{\pagestyle{empty}\cleardoublepage}
%\doublespacing
\newpage

\pagestyle{empty}
\pagenumbering{roman}
\newpage
\chapter*{\centering \large Resumen} 
\addcontentsline{toc}{chapter}{Resumen} % si queremos que aparezca en el índice
\markboth{Resumen}{Resumen} % encabezado

Este trabajo de investigación aborda el proceso crítico de clasificación de truchas. Actualmente, llevados a cabo de forma manual en la Laguna de Paucarcocha. Este trabajo busca reducir la mortandad y aumentar la eficiencia del proceso mediante el diseño conceptual de una máquina automatiza el proceso. Se sigue la metodología de diseño VDI. La selección del mejor diseño conceptual se realiza de entre tres propuestas.

%Considera los siguientes puntos:
%\begin{enumerate}
%	\item Desarrolle un único párrafo (200 a 300 palabras)
%	\item Escriba en tiempo verbal presente
%	\item El resumen debe contener información sobre:
%	\begin{itemize}
%		\item	- La justificación de la investigación
%		\item	- Los objetivos o hipótesis
%		\item	- La teoría o supuestos teóricos o metodológicos en la que se sustenta
%		\item	- El método o procedimiento realizado (de ser necesario)
%		\item	- Los resultados (de ser necesario)
%		\item	- La conclusión principal
%	\end{itemize}
%\end{enumerate}