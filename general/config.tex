\documentclass[12pt,a4paper,oneside]{report} % Formato tesis
\usepackage[english,spanish]{babel} % Asignar idioma por defecto
\selectlanguage{spanish} % Idioma por defecto x2
\usepackage[T1]{fontenc} % Aceptar caracteres con tíldes en el pdf generado
\usepackage[UTF8]{ctex}
\usepackage{times} %
\usepackage[utf8]{inputenc} % Aceptar caracteres con tíldes en Latex
\usepackage{amsmath} % Para mostrar de forma adecuada las ecuaciones
\usepackage{graphicx} % Para incluir imágenes 
\usepackage{array} % Para incluir imagenes en tablas https://bit.ly/2KXhIlu
\usepackage{float} % Para posicionar imágenes y tablas
\usepackage{longtable} % Para tablas que abarcan más de dos páginas
\usepackage{multicol} % Para poder partir en columnas 
\usepackage[refpages]{gloss} % Para enumerar páginas
\usepackage{anysize} % OBSOLETO, NECESITA SER CAMBIADO
\usepackage{bigstrut} % Para automatizar ingresado de datos en tabla
\usepackage{appendix} % Formato para títulos de los apéndices
\usepackage{lscape} % Modifica margenes y rota la página pero no el enumerado
\usepackage{pdflscape} % Cambiar orientación en el pdf
\usepackage{multirow} % Fácil manejo de filas
\usepackage{listings} % Sirve para que se compile el Latex
\usepackage{color} % Colores de fondo, texto y otros
\usepackage{setspace} % Espacio entre lineas
\usepackage{enumerate} % Estilo del contador de páginas
\usepackage{ragged2e} % Formato de parrafos, justiticar y demás
\usepackage{comment} % Hacer comentarios en latex
\usepackage{pslatex} % Tipo de fuente
\usepackage{apacite} % Citación en APA
\usepackage{fixltx2e} % Corrige bugs de Latex 3
\usepackage{caption} % Objetos flotantes en el documento
\usepackage[width=150mm,top=25mm,bottom=25mm]{geometry}
\usepackage{fancyhdr}
\usepackage{blindtext}
\usepackage[table,xcdraw]{xcolor}

\fancyhead{}  % Clears all page headers and footers

\setlength{\headheight}{15pt}
\pagestyle{fancy}
%\fancyhead[r]{\thepage}
%\fancyhead[l]{\fancyplain{\fancy}{\slshape\leftmark}}
%\fancyfoot[c]{center of the footer!}
\fancypagestyle{plain}{%
	\fancyhf{} % clear all header and footer fields
	\fancyhead[r]{\leftmark}
	\fancyfoot[c]{PUCP}
	\fancyfoot[r]{Pág. \thepage}
	\renewcommand{\headrulewidth}{0.4pt}
	\renewcommand{\footrulewidth}{0.4pt}}
\renewcommand{\headrulewidth}{0.4pt}
\renewcommand{\footrulewidth}{0.4pt}

\renewcommand{\chaptermark}[1]{\markboth{\itshape\chaptername~\thechapter}{}}
%\usepackage{fontspec}
%\setmainfont{ebgaramond}


\graphicspath{{images/}}
\captionsetup[table]{skip=10pt}
\bibliographystyle{apacite}
\renewcommand{\BOthers}[1]{et al.\hbox{}}
\renewcommand\bibname{Bibliografía}
%----------------------------------------------------------------------------------------
%	CONFIGURACION
%----------------------------------------------------------------------------------------
%\marginsize{2.5cm}{2.5cm}{2.5cm}{2.5cm}
\renewcommand*{\contentsname}{Tabla de contenidos}
\renewcommand{\listtablename}{Índice de tablas}
\renewcommand*{\listfigurename}{Índice de figuras}
\renewcommand{\baselinestretch}{1.0}
\renewcommand{\appendixname}{Anexos}
\renewcommand{\appendixtocname}{Anexos}
\renewcommand{\appendixpagename}{Anexos}
\renewcommand{\thetable}{\arabic{chapter}.\arabic{table}}
\renewcommand*{\tablename}{Tabla}
\renewcommand*{\chaptername}{Capítulo}
\renewcommand*{\thechapter}{\Roman{chapter}}
\renewcommand{\thesection}{\arabic{chapter}.\arabic{section}}
\renewcommand{\figurename}{Figura}
\renewcommand{\thefigure}{\arabic{chapter}.\arabic{figure}}
\renewcommand{\theequation}{\arabic{chapter}.\arabic{equation}}
\newcommand*\rot{\rotatebox{90}}

\captionsetup{justification=centering}